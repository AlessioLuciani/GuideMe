Sono state analizzate delle potenziali app concorrenti che rispondono ad alcuni dei bisogni
che abbiamo individuato:

\subsubsection*{App n\degree \ 1 - Roma Guida Turistica}
L'app prevede una mappa e una guida, funzionanti anche offline, e si basa sul gps.
La mappa è abbastanza dettagliata, infatti è possibile cercare vie,
negozi, attrazioni, hotel, bar e opere pubbliche.
È anche possibile impostare dei pin nella mappa creando una sorta di itinerario.
È presente una classifica di luoghi: cliccando su un luogo, si può mostrare sulla mappa,
vedere più foto del posto e vedere gli hotel o b\&b nelle vicinanze.
Alcuni luoghi hanno anche una breve descrizione presa da Wikipedia
(è possibile addirittura mostrare l’intera pagina di Wikipedia direttamente dall’app).
La guida è proprio rappresentata dalla semplice pagina di wikipedia, 
così come gli itinerari sono i luoghi che vengono aggiunti alle proprie liste.

\subsubsection*{App n\degree \ 2 - Quick Museum}
L'app è incentrata sui musei e la loro visita. Propone itinerari personalizzati all'interno dei musei e altri che permettono invece di raggiungerli dall'esterno (con tanto di mappa). Inoltre offre delle audio guide che permettono all'utente una migliore fruizione della visita, anche in più lingue.

\subsubsection*{App n\degree \ 3 - Monument Tracker}
L'app permette di programmare un itinerario, lasciando scegliere al software i monumenti e le attrazioni da visitare, in più ci proporrà dei quiz con le informazioni e le curiosità per ampliare le tue conoscenze e trovare siti nascosti da condividere con gli amici. Una funzionalità interessante è che invia notifiche quando ci si trova vicino ad un monumento interessante. Quest'app cerca di evitare l'utilizzo di una guida rendendo la visita una scoperta interattiva da parte dell'utente, che ne è il protagonista.