\documentclass[12pt]{article}\pagestyle{myheadings}
\renewcommand{\familydefault}{\rmdefault}

\usepackage[utf8]{inputenc} % usually not needed (loaded by default)
\usepackage{fontenc} 
\usepackage[italian,english]{babel}
\usepackage{enumitem}
\usepackage{gensymb}


\usepackage{hyperref}
\hypersetup{
    colorlinks=true, %set true if you want colored links
    linktoc=all,     %set to all if you want both sections and subsections linked
    linkcolor=black,  %choose some color if you want links to stand out
}
 
\setlength\parindent{0pt}

\usepackage{titling}
\usepackage{lipsum}  
\usepackage[a4paper,bindingoffset=0.2in,left=0.9in,right=0.9in,top=2in,bottom=1in,footskip=.25in]{geometry}
	
\newcounter{interview}
\setcounter{interview}{0}

\setcounter{secnumdepth}{0} % sections are level 1

\newcommand\Interview[1]{
   \stepcounter{interview}
	\markright{\normalfont Intervista n. \theinterview \ \ - \ \ #1}
}

\newcommand\invisiblesection[1]{%
  \refstepcounter{section}%
  \addcontentsline{toc}{section}{\protect{}#1}%
  \sectionmark{#1}}

\addto{\captionsenglish}{\renewcommand*{\contentsname}{Tabella dei contenuti}}

\setlist[enumerate]{label={\textbf{Domanda \arabic{enumi}:}},ref={Step \arabic{enumi}},leftmargin=*}

\date{}
%\setlength{\droptitle}{-5em}   % This is your set screw


\title{\textbf{\huge{Relazione GuideMe}}}
\author{Agosta, Belli, Luciani, Emili}


% ---------- END OF HEADER ---------- 

\begin{document}

\clearpage\maketitle

\thispagestyle{empty}
\clearpage

\thispagestyle{empty}
\tableofcontents

\clearpage

\section{Need Finding}


È stato fatto uno studio con il fine di individuare dei need relativi alla ricerca
di itinerari turistici. In particolare si è pensato al bisogno di potersi organizzare nel visitare città e
luoghi artistici in maniera spensierata e divertente. Molto spesso, quando ci si trova a 
visitare un nuovo posto, non si sa chiaramente quali sono le attrazioni e le opere che più
vale la pena di vedere. Inoltre può risultare difficile organizzare una gita che copra
diverse opere d'arte nella stessa giornata. Oppure, il tempo che si ha a disposizione potrebbe non
essere abbastanza per visitare tutto ciò che si ha in mente e quindi si deve giungere a dei compromessi.
Le opere che si incontrano nelle visite, inoltre, potrebbero non essere totalmente comprensibili
da un turista ignaro del loro significato, che quindi non potrebbe apprezzarle fino in fondo.
Può quindi risultare utile avere a disposizione un catalogo di itinerari selezionati e
garantiti da guide competenti che ci accompagnano nel percorso.

\clearpage

\section{Analisi delle App simili}

Sono state analizzate delle potenziali app concorrenti che rispondono ad alcuni dei bisogni
che abbiamo individuato:

\subsubsection*{App n\degree \ 1 - Roma Guida Turistica}
L'app prevede una mappa e una guida, funzionanti anche offline, e si basa sul gps.
La mappa è abbastanza dettagliata, infatti è possibile cercare vie,
negozi, attrazioni, hotel, bar e opere pubbliche.
E’ anche possibile impostare dei pin nella mappa creando una sorta di itinerario.
E’ presente una classifica di luoghi: cliccando su un luogo, si può mostrare sulla mappa,
vedere più foto del posto e vedere gli hotel o b\&b nelle vicinanze.
Alcuni luoghi hanno anche una breve descrizione presa da wikipedia
(è possibile addirittura mostrare l’intera pagina di wikipedia direttamente dall’app).
La guida è proprio rappresentata dalla semplice pagina di wikipedia, 
così come gli itinerari sono i luoghi che vengono aggiunti alle proprie liste, nulla di più.

\subsubsection*{App n\degree \ 2 - Visit a City}
L'app di per se permette di creare degli itinerari o usare quelli creati da altri utenti per una data città che vuoi visitare. Non è proprio uguale a quello che vogliamo fare noi perche non viene considerata nessun tipo di guida o gruppo.
Un pregio è che permette di usare questi itinerari sia online che offline salvando le mappe.

\subsubsection*{App n\degree \ 3 - Quick Museum}
L'app è incentrata sui musei e la loro visita. Propone degli itinerari personalizzati all'interno dei musei e tra di loro (è provvista di mappa interna dei musei).
Inoltre offre delle audio guide che immergono l'utente durante la visita multingua e con un linguaggio semplice

\subsubsection*{App n\degree \ 4 - Monument Tracker}
L'app permette di programmare un itinerario, lasciando scegliere al software i monumenti e le attrazioni da visitare, in più ci proporrà dei quiz con le informazioni e le curiosità per ampliare le tue conoscenze e trovare siti nascosti da condividere con gli amici. Inoltre una funzionalità interessante è che invia notifiche quando ci si trova vicino ad un monumento interessante e scoprire storie e aneddoti senza doversi rivolgere ad una guida.
Questa app cerca di escludere una guida per rendere piu una "scoperta" la visita.

\clearpage

\section{Conclusioni interviste}
Le interviste sono state condotte in luoghi diversi, variando il più possibile il target per cercare di trovare un campione che fosse il più verosimile possibile alla situazione reale. \\
Tutti i dati raccolti dalle interviste venivano annotati su blocco note e sono di seguito elencati.

\clearpage

E' stato condotto un questionario con lo scopo di verificare, su un ampio campione di persone, i need che sono stati individuati grazie alle interviste.\\

Il questionario è stato strutturato in modo tale da permettere all'utente in primo luogo di selezionare la lingua desiderata (a scelta tra italiano e inglese) e successivamente di navigare tra le sue sezioni. La scelta della doppia lingua nasce dall'idea di voler 
raggiungere più persone possibili al fine di validare le osservazioni relative alla fase delle interviste, con un'importante conseguenza: con l'aumentare della dimensione del campione il pubblico considerato tende ad essere più eterogeneo, ovvero diventa meno probabile considerare solo un target particolare. \\

Il questionario ha raggiunto infatti un pubblico maggiore di quello considerato nella fase delle interviste, arrivando a contare oltre 160 persone. \\

Inoltre prima di diffondere il questionario questo è passato per una fase di verifica, cosiddetto questionario pilota, sottoponendolo a 7 persone, di cui 5 italiane e 2 di lingua inglese.\\

Presentiamo di seguito le considerazioni più interessanti che sono risultate dai questionari. Si noti come le osservazioni considerano tutto il pubblico, quindi le percetuali si riferiscono, se non specificato, al pubblico italiano e quello inglese insieme.

\begin{itemize}

\item Il questionario ha per la maggior parte del suo pubblico i giovani, infatti questi rappresentano il 90\% del totale delle persone raggiunte.

\item Il 92\% del pubblico target preferisce le visite autonome a quelle con guide.

\item Oltre il 90\% del pubblico preferisce le visite autonome a quelle con guide.

\item Il 70\% del pubblico target preferisce organizzare gli itinerari per tempo.

\item Oltre l'80\% del pubblico non utilizza app del genere. Di questi, oltre il 70\% perchè non era a conoscenza della loro esistenza.

\end{itemize}

\newpage

\subsection{Grafici significativi}

\begin{center}
    \includegraphics[width=14cm]{../questionnaires/chart1.png}
\end{center}


\begin{center}
    \includegraphics[width=14cm]{../questionnaires/chart2.png}
\end{center}


\begin{center}
    \includegraphics[width=14cm]{../questionnaires/chart3.png}
\end{center}

\clearpage

\invisiblesection{Interviews}
%%%%%%%%%%%%%%%%%%%%%%%%%%%%%%%%%%%%%%%%%%%%%%%%%%%%%%

\Interview{Una ragazza (20-25 anni)}

\begin{enumerate}

\item Conosci o hai mai utilizzato app per l'organizzazione di itinerari o visite guidate? Se sì quali?

No, non ne conosco.

\item Hai mai avuto problemi nell'organizzazione di itinerari? Se sì quali?

Sì, ho avuto problemi nel gestire l'itinerario. Spesso mancava il tempo per visitare i luoghi. Gestirsi bene con il tempo è difficile è spesso ci si trova con poco tempo a disposizione per le visite. Per esempio mi sono trovata a volte che era troppo tardi per comprare i biglietti per alcune visite. In generale cerco di sfruttare il tempo al meglio.

\item Come prepari gli itinerari che vuoi seguire?

In questo caso cerco di sfruttare al meglio le risorse a disposizione, tenendo sempre a mente il problema del tempo.

\end{enumerate}

\clearpage
\Interview{Un ragazzo (20-25 anni)}

\begin{enumerate}

\item Conosci o hai mai utilizzato app per l'organizzazione di itinerari o visite guidate? Se sì quali?

Sì, conosco solo l'app PiratiInViaggio come app di itinerari, anche se non è ideale per specificare una richiesta o un itinerario che voglio fare. Spesso non ci sono le informazioni riguardo i posti da visitare. 

\item Hai mai avuto problemi nell'organizzazione di itinerari? Se sì quali?

No, non direttamente con gli itinerari. Però magari ne pianifico uno e considero troppo poco tempo, sia in senso positivo che negativo, quindi mi trovo a volte con troppo tempo libero oppure con tempo che non è sufficiente per la visita di alcuni luoghi che volevo visitare.

\end{enumerate}

\clearpage
\Interview{Un ragazzo (20-30 anni)}

\begin{enumerate}

\item Conosci o hai mai utilizzato app per l'organizzazione di itinerari o visite guidate? Se sì quali?

No, non ne conosco.

\item Hai mai avuto problemi nell'organizzazione di itinerari? Se sì quali?

Utilizzo spesso Google Maps per cercare i luoghi da visitare e spesso ho problemi nel reperire le info, non sempre sono accurate.

\item Come prepari gli itinerari che vuoi seguire?

Cerco sempre di pianificare il più possibile, così da arrivare preparato.

\item Hai mai avuto problemi con i metodi di pagamento?

No, perchè in genere pago sempre online per evitare problemi.

\end{enumerate}

\clearpage
\Interview{Un ragazzo (20-25 anni)}

\begin{enumerate}

\item Conosci o hai mai utilizzato app per l'organizzazione di itinerari o visite guidate? Se sì quali?

No, non ne conosco. In genere per fare queste cose cerco su Google.

\item Hai mai avuto problemi nell'organizzazione di itinerari? Se sì quali?

Le informazioni non sempre mi aiutano a scegliere i posti oppure non riesco a fare dei buoni itinerari a partire da questi.

\item Come prepari gli itinerari che vuoi seguire?

In genere vado nei luoghi che preferisco senza pensarci troppo, essenzialmente quando sono lì vedo quello che voglio fare.

\item Hai mai avuto problemi con i metodi di pagamento?

No, mai avuti.

\end{enumerate}

\clearpage
\Interview{Una ragazza (20 anni)}

\begin{enumerate}

\item Conosci o hai mai utilizzato app per l'organizzazione di itinerari o visite guidate? Se sì quali?

No, non ne ho mai usata nessuna e non le conosco personalmente.

\item Hai mai avuto problemi nell'organizzazione di itinerari? Se sì quali?

A volte cercando su internet mancano le informazioni di cui avrei bisogno.

\item Come prepari gli itinerari che vuoi seguire?

Cerco sempre di pianificare il più possibile, così da arrivare preparato.

\item Hai mai avuto problemi con i metodi di pagamento?

No, mai avuto problemi.

\end{enumerate}

\clearpage
\Interview{Una ragazza (25-30 anni)}

\begin{enumerate}

\item Conosci o hai mai utilizzato app per l'organizzazione di itinerari o visite guidate? Se sì quali?

Cerco su Internet oppure uso RomaToday per visitare Roma oppure uso Instagram per avere ispirazioni di luoghi da visitare.

\item Hai mai avuto problemi nell'organizzazione di itinerari? Se sì quali?

No, sono sempre molto autonoma nel percorso quindi mi adatto ai vari luoghi da visitare anche se devo poi camminare molto. In genere mi baso sui luoghi noti e poi passeggio lì intorno.

\item Come prepari gli itinerari che vuoi seguire?

Cerco sempre di pianificare il più possibile, così da arrivare preparato.

\item Hai mai avuto problemi con i metodi di pagamento?

No, mai avuto problemi.

\item Hai mai avuto problemi con l'organizzazione del tempo?

Spesso manca tempo, ma anche qui mi organizzo abbastanza bene.

\end{enumerate}

\clearpage
\Interview{Una ragazza (25-30 anni)}

\begin{enumerate}

\item Conosci o hai mai utilizzato app per l'organizzazione di itinerari o visite guidate? Se sì quali?

Conosco e utilizzo Google Arts, ma non è comunque troppo utile per queste cose.

\item Hai mai avuto problemi nell'organizzazione di itinerari? Se sì quali?

No, nessun problema.

\item Come prepari gli itinerari che vuoi seguire?

Studio storia dell'arte, quindi in genere so già cosa voglio visitare e non ho grandi problemi a muovermi.

\item Hai mai avuto problemi con i metodi di pagamento?

No, mai avuto problemi.

\end{enumerate}

\clearpage
\Interview{Una ragazza (20-25 anni)}

\begin{enumerate}

\item Conosci o hai mai utilizzato app per l'organizzazione di itinerari o visite guidate? Se sì quali?

Conosco Google Arts e qualche altra app simile.

\item Hai mai avuto problemi nell'organizzazione di itinerari? Se sì quali?

No, non ho mai avuto nessun problema.

\item Come prepari gli itinerari che vuoi seguire?

Conosco i luoghi che voglio visitare e organizzo gli itinerari di conseguenza in modo autonomo.

\item Hai mai avuto problemi con i metodi di pagamento?

No, mai avuto problemi.

\end{enumerate}

\clearpage
\Interview{Una ragazza (25 anni)}

\begin{enumerate}

\item Conosci o hai mai utilizzato app per l'organizzazione di itinerari o visite guidate? Se sì quali?

No, non ne conosco.

\item Hai mai avuto problemi nell'organizzazione di itinerari? Se sì quali?

No, in genere ho una lista di posti da visitare e li visito.

\item Come prepari gli itinerari che vuoi seguire?

Cerco su Google e navigo su internet.

\item Hai mai avuto problemi con i metodi di pagamento?

No, mai avuto problemi. In genere uso le app come i SaltaFila tramite internet per evitare lunghe code.

\item Hai mai avuto problemi con l'organizzazione del tempo?

Sì, e parto sapendo che potrei già averne, quindi mi regolo di conseguenza.

\end{enumerate}

\clearpage
\Interview{Una ragazza (20-30 anni)}

\begin{enumerate}

\item Conosci o hai mai utilizzato app per l'organizzazione di itinerari o visite guidate? Se sì quali?

No, spesso cerco su Internet.

\item Hai mai avuto problemi nell'organizzazione di itinerari? Se sì quali?

Spesso è difficile collegare i luoghi/edifici che vorrei visitare e sono in difficoltà. Sarebbe comodo avere un itinerario già pronto.

\item Hai mai avuto problemi con i metodi di pagamento?

No, mai avuto problemi.

\item Hai mai avuto problemi con l'organizzazione del tempo?

I problemi col tempo erano spesso dovuto alle file lunghe, spesso programmo prima le cose per evitare di buttarne altro.

\end{enumerate}

\clearpage
\Interview{Una ragazza (20-25 anni)}

\begin{enumerate}

\item Conosci o hai mai utilizzato app per l'organizzazione di itinerari o visite guidate? Se sì quali?

No, in genere vado su Internet per cercare i posti da visitare.

\item Hai mai avuto problemi nell'organizzazione di itinerari? Se sì quali?

No problemi no. In genere pianifico le visite in base anche alla compagnia con cui mi trovo. Quindi scelgo i posti in base ai nostri interessi.

\item Come prepari gli itinerari che vuoi seguire?

Spesso li cerco online.

\item Hai mai avuto problemi con i metodi di pagamento?

No, mai avuto problemi. In genere uso le app come i SaltaFila tramite internet per evitare lunghe code.

\item Hai mai avuto problemi con l'organizzazione del tempo?

Sì, molto spesso infatti ho perso visite a musei che volevo vedere perchè non avevo abbastanza tempo.

\end{enumerate}

\clearpage
\Interview{Un ragazzo (20-25 anni)}

\begin{enumerate}

\item Conosci o hai mai utilizzato app per l'organizzazione di itinerari o visite guidate? Se sì quali?

Utilizzo Google Maps per vedere e collegare i punti da visitare. Così ho anche le distanze tra i vari luoghi.

\item Hai mai avuto problemi nell'organizzazione di itinerari? Se sì quali?

No, mai avuti grossi problemi.

\item Come prepari gli itinerari che vuoi seguire?

Li organizzo io cercando i posti su Internet.

\item Hai mai avuto problemi con i metodi di pagamento?

No, mai avuto problemi.

\item Hai mai avuto problemi con l'organizzazione del tempo?

Usando Google Maps so le disanze e posso creare un piccolo itinerario in poco tempo che mi permette anche di stimare quanto tempo mi richiederà.

\end{enumerate}

\clearpage
\Interview{Una ragazza (25 anni)}

\begin{enumerate}

\item Conosci o hai mai utilizzato app per l'organizzazione di itinerari o visite guidate? Se sì quali?

No, non ne conosco.

\item Hai mai avuto problemi nell'organizzazione di itinerari? Se sì quali?

Spesso mi affido ad altri che li organizzano per me, però ho avuto comunque piccoli problemi.

\item Come prepari gli itinerari che vuoi seguire?

Li organizzano gli altri (miei amici).

\item Hai mai avuto problemi con i metodi di pagamento?

No, mai avuto problemi.

\item Hai mai avuto problemi con l'organizzazione del tempo?

Sì, a volte.

\item Hai mai ricorso alle guide per visite a questi posti?

Sì, quando ero più piccola ma non di recente.

\end{enumerate}

\clearpage
\Interview{Un ragazzo (20-25 anni)}

\begin{enumerate}

\item Conosci o hai mai utilizzato app per l'organizzazione di itinerari o visite guidate? Se sì quali?

Sì, spesso utilizzo Google Maps per organizzarli.

\item Hai mai avuto problemi nell'organizzazione di itinerari? Se sì quali?

A volte potevo perdermi dei luoghi che non pensavo fossero interessanti perchè le informazioni che avevo trovato non mi avevano convinto.

\item Come prepari gli itinerari che vuoi seguire?

Li organizzo aiutandomi con dei siti Internet che li pubblicizzano.

\item Hai mai avuto problemi con i metodi di pagamento?

No, in genere non ho problemi.

\item Hai mai avuto problemi con l'organizzazione del tempo?

Sì, a volte non mi bastava per visitare luoghi che volevo vedere. Altre volte invece mi sono proprio dimenticato di averli inclusi nell'itinerario.

\item Hai mai ricorso alle guide per visite a questi posti?

Dipende dal tipo di visita, però si a volte le ho utilizzate.

\end{enumerate}

\clearpage
\Interview{Un ragazzo (20-25 anni)}

\begin{enumerate}

\item Conosci o hai mai utilizzato app per l'organizzazione di itinerari o visite guidate? Se sì quali?

Sì, a volte uso app che parlano di arte/turismo per avere informazioni su posti da visitare.

\item Hai mai avuto problemi nell'organizzazione di itinerari? Se sì quali?

No, non particolarmente. In genere mi documento abbastanza bene cercando da varie fonti ma spesso è difficile trovare le informazioni corrette tra le tante disponibili.

\item Come prepari gli itinerari che vuoi seguire?

Li organizzo io con l'aiuto di altri amici se pianifico di andarci con loro.

\item Hai mai avuto problemi con i metodi di pagamento?

No, mai avuto problemi.

\item Hai mai avuto problemi con l'organizzazione del tempo?

Sì, a volte.

\item Hai mai ricorso alle guide per visite a questi posti?

Sì, a scuola sicuramente e poi anche una volta quando sono stato a Praga eravamo lì in gruppo e abbiamo deciso di prendere una guida per illustrarci il castello.

\end{enumerate}

\clearpage
\Interview{Una ragazza (20-25 anni)}

\begin{enumerate}

\item Conosci o hai mai utilizzato app per l'organizzazione di itinerari o visite guidate? Se sì quali?

L'intevistata ha partecipato a visite turistiche e afferma che nella maggior parte dei casi non è lei che organizza l'itinerario ma preferisce farlo fare ad altri componenti del gruppo, perche ritiene che sia noioso dover cercare e poi organizzare un percorso tra le varie attrazioni.
Per quanto riguarda le app non ne conosco nessuna.

\item Hai mai avuto problemi nell'organizzazione di itinerari? Se sì quali?

Si, nelle rare volte che l'intervistata ha organizzato l'itinerario ha trovato problemi nel reperire informazioni \(o trovarne di sbagliate\) riguardo ai luoghi da visitare.

\item Come prepari gli itinerari che vuoi seguire?

Principalmente cerco su internet quali sono i luoghi consigliati o sui siti che propongono delle liste di luoghi.

\item TI capita di andartene da un posto e scoprire che non hai visitato un luogo interessante?

Si è capitato di non aver visitato un posto un po piu particolare e non molto conosciuto.

\item Hai mai avuto problemi con i metodi di pagamento?

No, Non ho mai avuto problemi, anzi se possibile preferisco sempre acquistare il biglietto alla biglietteria del luogo che visito, cosi mi rimane anche il biglietto.

\item Hai mai avuto problemi con l'organizzazione del tempo?

Sì, quasi sempre non conoscendo bene la citta succede di gestire male il tempo a disposizione, talvolta anche a causa della disorganizzazione dei luoghi visitati, e quindi si devono scartare dei luoghi dall'itinerario programmato.

\item Hai mai ricorso alle guide per visite a questi posti?

No, uno dei requisiti fondamentali è spendere il meno possibile, tanto alcune informazioni si possono trovare su internet.


\end{enumerate}

\clearpage
\Interview{Una ragazza (20-25 anni)}

\begin{enumerate}

\item Conosci o hai mai utilizzato app per l'organizzazione di itinerari o visite guidate? Se sì quali?

Si, ho provato una volta ad utilizzare \emph{amuseapp} ma ho subito lasciato perdere perche non riuscivo a selezionare l'italiano.

\item Hai mai avuto problemi nell'organizzazione di itinerari? Se sì quali?

In genere lascio organizzare ai miei amici le visite da fare, ma quando capita di farlo a me di solito cerco su internet i posti stimando piu o meno quanto starò in ognuno di essi.
Uno dei problemi che mi capita spesso è di prevedere troppo poco tempo per un museo o un luogo e poi devo fare le corse per completare l'itinerario.

\item Come prepari gli itinerari che vuoi seguire?

Principalmente cerco su  google quali posti ci sono in una città e dopo faccio il percorso.

\item Hai mai avuto problemi con i metodi di pagamento?

No, accettano quasi sempre la carta soprattutto in posti esteri.

\item Hai mai avuto problemi con l'organizzazione del tempo?

Sì, come dicevo spesso capita di pensare che un certo tempo basti per visitare un posto e invece non basta.

\item Hai mai ricorso alle guide per visite a questi posti?

No, una volta ho preso uno di quei bus turistici, ma solitamente preferisco andare per la città autonomamente.


\end{enumerate}

%%%%%%%%%%%%%%%%%%%%%%%%%%%%%%%%%%%%%%%%%%%%%%%%%%%%%%

% Restore normal (blank) text at the top of each page.
\markright{}

\clearpage

\section{Revisioni}

\begin{table*}[!h]
\centering
\begin{tabular}{|@{}p{3cm}@{}|@{}p{4.5cm}@{}|@{}p{2cm}@{}|}
 \hline
 \ Numero & \ Descrizione & \ Data \\
 \hline
 \ 1 &  &  \\
 \hline
 \ 2 &  &  \\
 \hline
\end{tabular}
\end{table*}

\end{document}