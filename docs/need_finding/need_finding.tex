

È stato fatto uno studio con il fine di individuare dei need relativi alla ricerca
di itinerari turistici. In particolare si è pensato al bisogno di potersi organizzare nel visitare città e
luoghi artistici in maniera spensierata e divertente. Molto spesso, quando ci si trova a 
visitare un nuovo posto, non si sa chiaramente quali sono le attrazioni e le opere che più
vale la pena di vedere. Inoltre può risultare difficile organizzare una gita che copra
diverse opere d'arte nella stessa giornata. Oppure, il tempo che si ha a disposizione potrebbe non
essere abbastanza per visitare tutto ciò che si ha in mente e quindi si deve giungere a dei compromessi.
Le opere che si incontrano nelle visite, inoltre, potrebbero non essere totalmente comprensibili
da un turista ignaro del loro significato, che quindi non potrebbe apprezzarle fino in fondo.
Può quindi risultare utile avere a disposizione un catalogo di itinerari selezionati e
garantiti da guide competenti che ci accompagnano nel percorso.

\subsection*{Competitors analysis}

Sono state analizzate delle potenziali app concorrenti che rispondono ad alcuni dei bisogni
che abbiamo individuato:

\subsubsection*{Competitor 1}
L'app contiene una collezione di itinerari, guide testuali e audioguide che ti accompagnano
alla scoperta di monumenti e percorsi suggestivi nel mondo.
Tutte queste guide sono precaricate da persone che conoscono bene il posto.
L’app da la possibilità di mostrare un itinerario sulla mappa, che può essere seguito
mentre si ascolta l’audioguida.
Si può scegliere un itinerario a piacere oppure utilizzare la modalità camminata libera,
in cui, non appena si trova un punto che è fornito di una spiegazione testuale, o vocale,
se ne può usufruire.
L’app però non permette alle guide in persona di organizzarsi con i visitatori per degli
itinerari programmati.

\subsubsection*{Competitor 2}
L'app prevede una mappa e una guida, funzionanti anche offline, e si basa sul gps.
La mappa è abbastanza dettagliata, infatti è possibile cercare vie,
negozi, attrazioni, hotel, bar e opere pubbliche.
E’ anche possibile impostare dei pin nella mappa creando una sorta di itinerario.
E’ presente una classifica di luoghi: cliccando su un luogo, si può mostrare sulla mappa,
vedere più foto del posto e vedere gli hotel o b\& b nelle vicinanze.
Alcuni luoghi hanno anche una breve descrizione presa da wikipedia
(è possibile addirittura mostrare l’intera pagina di wikipedia direttamente dall’app).
La guida è proprio rappresentata dalla semplice pagina di wikipedia, 
così come gli itinerari sono i luoghi che vengono aggiunti alle proprie liste, nulla di più.
