

È stato fatto uno studio con il fine di individuare dei need relativi alla ricerca
di itinerari turistici. In particolare si è pensato al bisogno di potersi organizzare nel visitare città e
luoghi artistici in maniera spensierata e divertente. Molto spesso, quando ci si trova a 
visitare un nuovo posto, non si sa chiaramente quali sono le attrazioni e le opere che più
vale la pena di vedere. Inoltre può risultare difficile organizzare una gita che copra
diverse opere d'arte nella stessa giornata. Oppure, il tempo che si ha a disposizione potrebbe non
essere abbastanza per visitare tutto ciò che si ha in mente e quindi si deve giungere a dei compromessi.
Le opere che si incontrano nelle visite, inoltre, potrebbero non essere totalmente comprensibili
da un turista ignaro del loro significato, che quindi non potrebbe apprezzarle fino in fondo.
Può quindi risultare utile avere a disposizione un catalogo di itinerari selezionati e
garantiti da guide competenti che ci accompagnano nel percorso.

\clearpage

\section{Analisi delle App simili}

Sono state analizzate delle potenziali app concorrenti che rispondono ad alcuni dei bisogni
che abbiamo individuato:

\subsubsection*{App n\degree \ 1 - Roma Guida Turistica}
L'app prevede una mappa e una guida, funzionanti anche offline, e si basa sul gps.
La mappa è abbastanza dettagliata, infatti è possibile cercare vie,
negozi, attrazioni, hotel, bar e opere pubbliche.
E’ anche possibile impostare dei pin nella mappa creando una sorta di itinerario.
E’ presente una classifica di luoghi: cliccando su un luogo, si può mostrare sulla mappa,
vedere più foto del posto e vedere gli hotel o b\&b nelle vicinanze.
Alcuni luoghi hanno anche una breve descrizione presa da wikipedia
(è possibile addirittura mostrare l’intera pagina di wikipedia direttamente dall’app).
La guida è proprio rappresentata dalla semplice pagina di wikipedia, 
così come gli itinerari sono i luoghi che vengono aggiunti alle proprie liste, nulla di più.

\subsubsection*{App n\degree \ 2 - Visit a City}
L'app di per se permette di creare degli itinerari o usare quelli creati da altri utenti per una data città che vuoi visitare. Non è proprio uguale a quello che vogliamo fare noi perche non viene considerata nessun tipo di guida o gruppo.
Un pregio è che permette di usare questi itinerari sia online che offline salvando le mappe.

\subsubsection*{App n\degree \ 3 - Quick Museum}
L'app è incentrata sui musei e la loro visita. Propone degli itinerari personalizzati all'interno dei musei e tra di loro (è provvista di mappa interna dei musei).
Inoltre offre delle audio guide che immergono l'utente durante la visita multingua e con un linguaggio semplice

\subsubsection*{App n\degree \ 4 - Monument Tracker}
L'app permette di programmare un itinerario, lasciando scegliere al software i monumenti e le attrazioni da visitare, in più ci proporrà dei quiz con le informazioni e le curiosità per ampliare le tue conoscenze e trovare siti nascosti da condividere con gli amici. Inoltre una funzionalità interessante è che invia notifiche quando ci si trova vicino ad un monumento interessante e scoprire storie e aneddoti senza doversi rivolgere ad una guida.
Questa app cerca di escludere una guida per rendere piu una "scoperta" la visita.