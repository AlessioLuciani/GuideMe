\section{Questionari}

Il questionario è stato strutturato in modo tale da permettere all'utente in primo luogo di selezionare la lingua desiderata (a scelta tra italiano e inglese) e successivamente di navigare tra le sue sezioni. La scelta della doppia lingua nasce dall'idea di voler 
raggiungere più persone possibili al fine di validare le osservazioni relative alla fase delle interviste, con un'importante conseguenza: con l'aumentare della dimensione del campione il pubblico considerato tende ad essere più eterogeneo, ovvero diventa meno probabile considerare solo un target particolare. \\

Il questionario ha raggiunto infatti un pubblico maggiore di quello considerato nella fase delle interviste, arrivando a contare oltre 160 persone. \\

Inoltre prima di diffondere il questionario questo è passato per una fase di verifica, cosiddetto questionario pilota, sottoponendolo a 7 persone, di cui 5 italiane e 2 di lingua inglese.\\

Presentiamo di seguito le considerazioni più interessanti che sono risultate dai questionari. Si noti come le osservazioni considerano tutto il pubblico, quindi le percetuali si riferiscono, se non specificato, al pubblico italiano e quello inglese insieme.

\begin{itemize}

\item Il questionario ha per la maggior parte del suo pubblico i giovani, infatti questi rappresentano il 90\% del totale delle persone raggiunte.

\item Il 92\% del pubblico target preferisce le visite autonome a quelle con guide.

\item Il 70\% del pubblico target preferisce organizzare gli itinerari per tempo.

\end{itemize}


Inoltre nel modulo è presente una sezione di analytics per permetterci di studiare meglio le risposte in base alla tipologia di utente.\\

\subsection{Conclusioni}

Analizzando i risultati del questionario su circa 160 utenti abbiamo potuto osservare che questionari hanno confermato i need emersi durante le interviste. Abbiamo potuto confermare quanto osservato nello step 3 del Process, infatti:

\begin{itemize}

	\item gli utenti preferiscono visitare autonomamente le città e non avvalersi di una guida.
	\item gli utenti preferiscono organizzare l'itinerario prima di arrivare nella città
	\item la maggior parte degli utenti non usa app di questo genere perchè non ne è a conoscenza, mentre una parte pensa che fare gli itenari per conto proprio sia uguale.
	% risolvere problema con la domanda preferisci creare da solo gli itinerari che è in contraddizione con le interviste.

\end{itemize}