\section{Questionari}

Le varie risposte raccolte con i questionari hanno confermato il nostro cambio di idea//
Gli utenti target rappresentano il 90percento di tutti gli intervistati//
Gli utenti target preferiscono visite autonome a quelle con guide (92percento)//
Il 70percente dei ragazzi (ita+eng) preferiscono organizzare itinerari per tempo//

In questo step abbiamo utilizzato i questionari per validare le informazioni e i need reperiti durante le interviste.
Questa operazione è stata necessaria per capire se effettivamente le persone avevano necessità delle cose scoperte e non avevamo intervistato tutte persone con la stessa opinione ma che non rifletteva i reali need.\\
Per questo abbiamo creato un questionario basato sulle domande fatte nelle interviste anche se un po piu mirate piuttosto che in generale \(anche perche nelle interviste è stato dato spazio all'intervistato di divagare e fare osservazioni talvolta non pertinenti\).\\
Il questionario è diviso in 2 parti una in italiano ed una in inglese cosi che anche chi non conosce bene l'italiano ha modo di fornire la sua opinione.\\
Il questionario ha prima subito una fase di verifica \(questionario pilota\) sottoponendolo a circa 7 persone di cui 5 italiane e 2 inglesi per accertarsi che le domande/risposte fossero corrette e non avessero ambiguita o incomprensioni.\\
Una volta apputato questo abbiamo condiviso il link a quante piu persone potevamo cosi da avere molte risposte e quindi una conferma dei need piu affidabile. Nel modulo è stata prevista una sezione di analytics in modo che anche se persone fuori dal target avessero risposto al questionario, saremmo comunque riusciti ad isolare i casi utili da quelli non utili; inoltre questa distinzione ci dava la possibilità di capire se un'altro target aveva altri need oppure no.\\

\subsection{Conclusioni}
I questionari hanno confermato i need riscontrati dagli utenti intervistati e hanno anche confermato il nostro cambio di idea infatti:\\

Nota: gli utenti target rappresentano il 90\% delle risposte al questionario\\ 

\begin{itemize}

	\item gli utenti preferiscono visitare autonomamente le città e non avvalersi di una guida.
	\item gli utenti preferiscono organizzare l'itinerario prima di arrivare nella città
	\item la maggior parte degli utenti non usa app di questo genere perche non è a conoscenza, mentre una parte pensa che fare gli itenari per conto proprio sia uguale.
	% risolvere problema con la domanda preferisci creare da solo gli itinerari che è in contraddizione con le interviste.