\subsection{\emph{Step 1 - L'idea}}
Il processo di sviluppo dell'applicazione parte dall'idea di voler realizzare un sistema che avvicini le guide ai clienti.\\
A tal proposito volevamo un sistema che permettesse alle guide di pubblicare itinerari personalizzati con tutte le relative informazioni (orari, prezzi, monumenti visitati...).\\
L'app avrebbe permesso quindi agli utenti di acquistare un \emph{ticket} per partecipare ad un itinerario condiviso con altri utenti (anch'essi in possesso del ticket). Creando tale gruppo potevano trarne vantaggio non solo dal punto di vista economico ma l'intero processo di scelta della guida si poteva far comodamente dal proprio smartphone.
Da questa idea di base venivano identificati dei need che dovevano poi esser verificati tramite delle interviste; in prima battuta i need evidenziati sono stati:
\begin{itemize}
	\item Assenza di informazioni riguardanti luoghi pubblici o monumenti.
	\item Trovare quali monumenti visitare in una città sconosciuta 
	\item Come trovare una guida turistica una volta arrivati in una particolare città
	\item Gestire nel miglior modo il tempo a disposizione per visitare i posti disponibili
\end{itemize}
\subsection{\emph{Step 2 - Analisi delle app simili}} %da rivedere meglio
Dopo aver definito la nostra idea e quali need potevano esserci, abbiamo fatto un'analisi delle app simili. Abbiamo preso in considerazione le app più conosciute che trattano di itinerari con guide e visite di luoghi culturali per capire cosa ci fosse di simile già in commercio.\\
Durante questa fase non sono state trovate app che svolgessero, in tutto o in parte, le funzionalità che previste dalla nostra app.
\subsection{\emph{Step 3 - Interviste} \& \emph{conseguenze}}
Successivamente sono stati intervistati dei candidati previsti dal target scelto ed è risultato che l'idea proposta non rifletteva i need delle persone. Sono emersi i seguenti need:
\begin{itemize}
	\item La totalità degli intervistati ha affermato di non voler usufruire di guide professioniste, principalmente per una questione economica.
	\item Gli utenti hanno confermato che preferiscono non dover perdere tempo nel ricercare i luoghi da visitare per creare un itinerario. Inoltre alcuni intervistati affermano di trovare i monumenti da visitare come singoli, ma in quanto tali è difficile organizzarli in un itinerario.
	\item È stato confermato che spesso nell'organizzazione degli itinerari viene mal gestito il tempo a disposizione per le visite.
\end{itemize}
Dopo aver analizzato le interviste, abbiamo capito che uno dei punti centrali della nostra idea non è in realtà necessità del target considerato. Quindi abbiamo deciso di modificare l'app escludendo le guide e sostituendole con gli utenti stessi: si passa a una nuova idea che consente agli utenti di consultare itinerari  pubblicati da altri utenti che già hanno fatto quel particolare itinerario in una delle loro esperienze di visita. \\
L'idea è che così facendo siamo in grado di mantenere la fruizione degli itinerari da parte degli utenti, rendendola però totalmente gratuita. \\
Inoltre per mantenere itinerari di qualità è previsto un sistema di recensioni che aiuti gli utenti: coloro che hanno già seguito l'itinerario in passato lasciano un feedback sulla base dell'esperienza che hanno avuto, consentendo agli altri di farsi un'idea. Tramite queste informazioni si rende possibile realizzare un ranking degli itinerari disponibili in modo da facilitare l'utente nella scelta.\\
% questionario pilota
%questionario
%identificazione tasks
%storyboard
%prototyping (loop)



