\subsection{\emph{Step 1 - Idea}}
Il processo di sviluppo dell'applicazione parte dall'idea di voler creare un sistema che avvicini le guide ai clienti.\\
A tal proposito volevamo un sistema che permettesse alle guide di pubblicare degli itinerari personalizzati con annesse tutte le informazioni su di essi (orari, prezzi, monumenti visitati...).\\
L'app avrebbe permesso poi agli utenti di comprare un "ticket" per partecipare ad un itinerario condiviso con altri utenti (anch'essi in possesso del ticket) cosi da creare un gruppo di visitatori traendone vantaggio non solo dal punto di vista economico ma anche dalla semplificazione nel poter scegliere la guida comodamente dal proprio smartphone.
Su questa idea di base sono stati identificati dei nostri need che poi dovevano essere verificati tramite interviste; in prima battuta i need evidenziati erano:
\begin{itemize}
	\item Assenza di informazioni riguardanti luoghi pubblici o monumenti.
	\item Trovare quali monumenti visitare in una città non conosciuta
	\item Come trovare una guida turistica una volta arrivati in una particolare città
	\item Gestire nel miglior modo il tempo a disposizione per visitare i posti disponibili
\end{itemize}
\subsection{\emph{Step 2 - Competitors analysis}} %da rivedere meglio
Dopo aver definito la nostra idea e quali need potevano esserci, abbiamo fatto un'analisi delle app simili, prendendo in considerazione le app piu conosciute che trattano di guide con itinerari e visite di luoghi culturali per capire cosa ci fosse di simile gia in commercio.\\
Durante questa fase non sono state evidenziate app che svolgessero in totale o in parte le funzionalità che avrebbe svolta la nostra app.
\subsection{\emph{Step 3 - Interviews \& consequences}}
Successivacmente sono state fatte le interviste a dei candidati presi secondo il target ed è stato evidenziato da queste che l'idea proposta non rifletteva i need delle persone ossia:
\begin{itemize}
	\item La totalità degli intervistati ha affermato di non voler usufruire di guide professioniste, maggiormente per una questione economica.
	\item \'E stato confermato da parte degli utenti che preferirebbero non dover perdere tempo nel ricercare i luoghi da visitare per creare un itinerario, inoltre alcuni intervistati affermano di trovare i monumenti da visitare ma sono solo liste statiche non adattabili a vincoli di tempo o spazio dell'utente.
	\item \'E stato confermato che spesso nell'organizzazione degli itinerari viene mal gestito il tempo a disposizione per le visite
\end{itemize}
Dopo aver analizzato le interviste, abbiamo capito che uno dei punti centrali della nostra idea di app non è una necessità del target da noi designato; quindi abbiamo deciso di modificare l'app escludendo le guide sostituendole con gli utenti stessi, passando alla nuova idea che consentirà agli utenti di consultare degli itinerari, ma questi ultimi, saranno pubblicati da altri utenti che hanno fatto quel particolare itinerario in una delle loro esperienze di visita. Visto che per i visitatori uno dei problemi era di base economico e allo stesso tempoidentificavano un problema nel trovare informazioni sui luoghi, cosi facendo manteniamo gli itinerari con tutte le informazioni a loro relative, ma la consultazione di tali itinerari sarà totalmente gratuita.\\
Inoltre per garantire un'affidabilità degli itinerari è previsto un sistema di recensione da parte di chi ha seguito l'itinerario in modo che chi volesse utilizzarlo avrebbe gia un'idea basata sull'esperienza di altri utenti. Tramite queste informazioni riusciamo anche a fare un ranking tra gli itinerari disponibili in modo da facilitare l'utente nella scelta.\\
% questionario pilota
%questionario
%identificazione tasks
%storyboard
%prototyping (loop)



