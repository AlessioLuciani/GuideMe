I prototipi cartacei sono stati resi interattivi utilizzando la piattaforma \textbf{Marvel App} e sono interamente fruibili sia da web che da app per smartphone. I prototipi sono disponibili sia per sistema operativo Android che per iOS.

\begin{itemize}

\item{\makebox[4.7cm]{link al prototipo Android:\hfill} \href{https://marvelapp.com/103e0g89}{\emph{https://marvelapp.com/103e0g89}}}
\item{\makebox[4.7cm]{link al prototipo iOS:\hfill} \href{https://marvelapp.com/5f6h8b2}{\emph{https://marvelapp.com/5f6h8b2}}}

\end{itemize}

\subsection{Descrizione degli scenari}

\begin{itemize}

\item \textbf{Navigazione:} Immagina di aver avviato un itinerario che parte dal Colosseo e ti sei recato di fronte ad esso. Hai interesse a leggere una descrizione del Colosseo e ad ascoltarla via audio. Vuoi vedere le altre tappe dell’itinerario che ti aspettano dopo il Colosseo. Vuoi selezionare velocemente la tappa dove ti trovi geograficamente in questo momento (sempre il Colosseo).

\item \textbf{Ricerca con filtri:} Ti trovi a casa tua e vuoi organizzare una visita dei luoghi all’aperto di Roma. Dopo aver fatto una ricerca per parola chiave vorresti poter applicare dei filtri su durata e lunghezza del percorso che ti piacerebbe seguire.

\item \textbf{Dettaglio itinerario:} Immagina di stare organizzando una visita  e di trovare il percorso perfetto per te, vuoi scoprire tutte le informazioni su quel percorso. Magari ti piace a tal punto da volerlo aggiungere ai preferiti e successivamente avviarlo.

\item \textbf{Recensione itinerario:} Immagina di aver già seguito in passato un itinerario con quest'app e ora che ti sei fatto un'idea vuoi condividere la tua esperienza con gli altri utenti. Vuoi aggiungere i dettagli sulla tua esperienza, magari condividendo foto scattate durante il tragitto.

\item \textbf{Crea nuovo itinerario:} Immagina di voler creare un nuovo itinerario per permettere ad altre persone di seguirlo. Vuoi condividere dettagli e curiosità circa il significato che ha per te quell'itinerario, aggiungendo descrizioni e informazioni utili circa il tempo richiesto per percorrerlo.

\end{itemize}

\subsection{Approccio utilizzato}

Ciascun task è stato testato su gruppi di 4-5 persone e l'approccio al protoyping scelto è stato di tipo evolutivo. Ad ogni iterazione vengono proposte modifiche al fine di migliorare il prototipo e risolvere i problemi evidenziati nella precedente versione. \\
La modalità di valutazione utilizzata è il \emph{Think Aloud}.

\clearpage

% ========================================================

\subsection{Iterazioni e modifiche apportate}

In questa sezione vengono descritti gli aspetti negativi e positivi riscontrati nelle varie versioni e le modifiche apportate per risolvere i problemi sollevati nelle precedenti.
\subsubsection{Versione 1}

\begin{itemize}

\item \textbf{Navigazione:} I pin non sono cliccabili, pulsante posizione è fuorviante come pure la freccia in basso.

\item \textbf{Ricerca con filtri:} Alcuni trovano di scarsa utilità il pulsante \emph{Cerca} e vorrebbero far partire l'itinerario dalla tastiera. Tuttavia il pulsante filtraggio è ben posizionato ed identificabile, la barra di ricerca immediatamente identificata e le informazioni esterne ben comprese, come pure il pulsante \emph{Aggiungi ai preferiti}.

\item \textbf{Dettaglio itinerario:} Alcuni utenti hanno difficoltà nel capire a cosa si riferisce il range di prezzo, manca il rating dell’itinerario e
le recensioni. Il pulsante \emph{Aggiungi ai preferiti} è stato perfettamente identificato, come pure il pulsante \emph{Avvia}, la descrizione e tutte le icone ad eccezione di quella del prezzo che viene compresa da alcuni ma non da tutti.

\item \textbf{Recensione itinerario:} L'utente vorrebbe essere in grado di avere informazioni riguardo l'itinerario di interesse. Inoltre l'hamburger menù non è cliccabile, come pure le icone per aggiungere le foto. Tutte le componenti sono state immediatamente identificate.

\item \textbf{Crea nuovo itinerario:} Alcuni utenti non hanno compreso il significato del pulsante \emph{Fatto} in basso, altri hanno frainteso il campo \emph{Nome} relativo all'itinerario col nome delle singole tappe ed infine, seppur non ci siano stati problemi con la comprensione del riepilogo, molti lo hanno trovato di dubbia utilità. Molti utenti hanno trovato comoda la disposizione delle tappe in una lista in quanto simile all'effetto ottenuto su Google Maps. Inoltre non ci sono stati problemi nella ricerca.

\end{itemize}

\subsubsection{Versione 2}

\begin{itemize}

\item \textbf{Navigazione:} Ora i pin sono cliccabili e portano alla tappa cliccata, è stata aggiunta una figura circolare blu che indica la posizione attuale. È stata sostituita la freccia con foglio ed infine su Android è stata aggiunto un Floating Action Button per indicare la posizione, che è stato ben identificato. Tuttavia è stato rilevato dagli utenti che il pulsante play è fuorviante, gli swipe a destra e sinistra non intuitivi e il pulsante di localizzazione prossima tappa anch'esso fuorviante.

\item \textbf{Ricerca con filtri:} È stata aggiunta la possibilità di ricercare tramite l’invio dalla tastiera, trovata molto comoda e rimosso il pulsante \emph{Cerca}. La ricerca è diventa più semplice da utilizzare secondo alcuni utenti.

\item \textbf{Dettaglio itinerario:} Sono state aggiunte le recensioni in fondo alla pagina che sono state ben identificate da tutti gli utenti e aggiunta mappa al posto dell’immagine. Entrambe sia la mappa che l'icona relativa al rating dell'itinerario sono state comprese da tutti.

\textbf{\emph Nota:} La fascia di prezzo è stata lasciata anche durante la seconda iterazione per continuare a testare la sua comprensione da parte degli utenti, riportando però lo stesso risultato ed è stata quindi rimossa dalla pagina di dettaglio.

\item \textbf{Recensione itinerario:} Sono stati resi cliccabili i pulsanti ed è stato aggiunto il pulsante \emph{Recensisci}, perfettamente identificato da tutti gli utenti. Tuttavia molti utenti lamentano la mancanza di un'informazione che gli dica quando è stato visitato l'itinerario.

\item \textbf{Crea nuovo itinerario:} Il pulsante fatto è stato sostituito dal Floating Action Button che ha icona \emph{Done}. Tuttavia persiste il problema del nome dell’itinerario confuso con quello della tappa da visitare. Inoltre l'utente ha trovato strano trovarsi di fronte a un riepilogo dopo la creazione di un itinerario. Il FAB per completare l'aggiunta dell'itinerario è stato perfettamente identificato.

\end{itemize}

\subsubsection{Versione 3}

\begin{itemize}

\item \textbf{Navigazione}: Il pulsante play/pause è stato sostituito col man speaking/stop, gli swipe da icone con frecce a destra e sinistra e il pulsante di localizzazione prossima tappa col pulsante \emph{Nearby}.

\item \textbf{Dettaglio itinerario:} Rimossa l'informazione relativa alla fascia di prezzo dell'itinerario in base a quanto precedemente osservato.

\item \textbf{Recensione itinerario:} È stata aggiunta la data di visita che è stata chiaramente identificata da tutti gli utenti, inoltre la recensione è possibile solo se l'itinerario è stato visitato dato che non vengono mostrati altrimenti, comportamento condiviso da tutti.

\item \textbf{Crea nuovo itinerario:} Sono stati aggiunti suggerimenti alle textfield per rimuovere ogni ambiguità nel significato del campo \emph{Nome}, la label che portava prima la stringa \emph{Nome} è stata sostituita con la stringa \emph{Titolo}. In effetti questi cambiamenti sono stati perfettamente compresi dagli utenti. Infine data la struttura della pagina schematica e intrinsecamente riepilogativa e dati soprattutto i feedback degli utenti si è deciso di rimuovere la pagina di riepilogo mostrando semplicemente un alert di conferma.

%La rimozione del riepilogo, oltre a non confondere l’utente, ha anche evitatoche quest’ultimo si infastidisse (cosa successa in intervista 1-versione 1)

\end{itemize}
