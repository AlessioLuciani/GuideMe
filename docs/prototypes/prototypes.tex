I prototipi cartacei sono stati resi interattivi utilizzando la piattaforma \textbf{Marvel App} e sono interamente fruibili sia da web che da app per smartphone. I prototipi sono disponibili sia per sistema operativo Android che per iOS.

\begin{itemize}

\item{\makebox[4.7cm]{link al prototipo Android:\hfill} \href{https://marvelapp.com/103e0g89}{\emph{https://marvelapp.com/103e0g89}}}
\item{\makebox[4.7cm]{link al prototipo iOS:\hfill} \href{https://marvelapp.com/5f6h8b2}{\emph{https://marvelapp.com/5f6h8b2}}}

\end{itemize}

\subsection{Descrizione degli scenari}

\begin{itemize}

\item \textbf{Navigazione:} Immagina di aver avviato un itinerario che parte dal Colosseo e ti sei recato di fronte ad esso. Hai interesse a leggere una descrizione del Colosseo e ad ascoltarla via audio. Vuoi vedere che altre tappe dell’itinerario ti aspettano dopo il Colosso. Vuoi selezionare velocemente la tappa dove ti trovi geograficamente in questo momento (sempre il Colosseo).

\item \textbf{Ricerca con filtri:} Ti trovi a casa tua e vuoi organizzare una visita dei luoghi all’aperto di Roma. Dopo aver fatto una ricerca per parola chiave vorresti poter applicare dei filtri su durata e lunghezza del percorso che ti piacerebbe seguire.

\item \textbf{Dettaglio itinerario:} Immagina di stare organizzando una visita  e di trovare il percorso perfetto per te, vuoi scoprire tutte le informazioni su quel percorso. Magari ti piace a tal punto da volerlo aggiungere ai preferiti e successivamente avviarlo.

\item \textbf{Recensione itinerario:} Immagina di aver già seguito in passato un itinerario con quest'app e ora che ti sei fatto un'idea vuoi condividere la tua esperienza con gli altri utenti. Vuoi aggiungere i dettagli sulla tua esperienza, magari condividendo foto scattate durante il tragitto.

\end{itemize}

\subsection{Approccio utilizzato}

Ciascun task è stato testato su gruppi di 4-5 persone e l'approccio al protoyping scelto è stato di tipo evolutivo. Ad ogni iterazione vengono proposte modifiche al fine di migliorare il prototipo e risolvere i problemi evidenziati nella precedente versione.

\clearpage

%================= NAVIGAZIONE ============================

\subsection{Navigazione}
\textbf{\emph Scenario}\\
Immagina di aver avviato un itinerario che parte dal Colosseo e ti sei recato di fronte
ad esso.\\
Hai interesse a leggere una descrizione del Colosseo e ad ascoltarla via audio.\\
Vuoi vedere che altre tappe dell’itinerario ti aspettano dopo il Colosso.\\
Vuoi selezionare velocemente la tappa dove ti trovi geograficamente in questo momento (sempre il Colosseo).
\newline
\textbf{\emph Versione 1}\\
\textbf{Aspetti negativi riscontrati:}
\begin{itemize}[label=-]

\item Pin non cliccabili
\item pulsante posizione fuorviante
\item freccia in basso fuorviante

\end{itemize}

\textbf{Risoluzioni:}
\begin{itemize}[label=-]

\item Rendere pin cliccabili (si salta alla tappa cliccata)
\item Aggiunta figura circolare blu, che indica la posizione attuale
\item Sostituzione freccia con foglio
\item Su android aggiungere floating action button con pulsante posizione al posto di pulsante in alto a destra

\end{itemize}

\textbf{\emph Versione 2}\\
\textbf{Aspetti negativi riscontrati:}
\begin{itemize}[label=-]

\item Pulsante play fuorviante
\item Swipe destra e sinistra non intuitivo
\item Pulsante localizzazione prossima tappa fuorviante

\end{itemize}

\textbf{Risoluzioni:}
\begin{itemize}[label=-]

\item Sostituzione pulsante play/pause con man speaking / stop
\item Sostituzione swipe destra e sinistra con frecce destra e sinistra
\item Sostituzione pulsante localizzazione prossima tappa con pulsante con scritta “prossima tappa”

\end{itemize}

%================= RICERCA FILTRATA ============================

\subsection{Ricerca Filtrata}
\textbf{\emph Scenario}\\
Immagina di star organizzando una visita dei luoghi all’aperto di Roma.\\
Dopo aver fatto una ricerca per parola chiave vorresti poter applicare dei filtri ad essa sulla durata e sulla lunghezza del percorso che ti piacerebbe fare.
\newline
\textbf{\emph Versione 1}\\
\textbf{Aspetti negativi riscontrati:}
\begin{itemize}[label=-]

\item utilità pulsante cerca, in alcuni test l’utente pensava che sarebbe partita subito la ricerca e non premeva il pulsante cerca

\end{itemize}

\textbf{Aspetti positivi riscontrati:}
\begin{itemize}[label=-]

\item pulsanti filtraggio ben posizionato ed identificati al primo colpo
\item pulsante ricerca subito identificato
\item ricerca effettuata correttamente
\item informazioni esterne ben comprese (anche icone)

\end{itemize}

\textbf{Risoluzioni:}
\begin{itemize}[label=-]

\item la ricerca parte direttamente tramite l’invio dalla tastiera, rimosso pulsante cerca.

\end{itemize}

\textbf{\emph Versione 2}\\
\textbf{Aspetti negativi riscontrati:}
\newline\newline
\textbf{Aspetti positivi riscontrati:}
\begin{itemize}[label=-]

\item più facilità nel inviare la ricerca

\end{itemize}

%================= DETTAGLIO ============================

\subsection{Dettaglio}
\textbf{\emph Scenario}\\
Immagina di star organizzando una visita e di aver trovato il percorso perfetto per te,
e di voler capire tutte le informazioni di quel percorso.\\
Inoltre ti piacerà talmente tanto da volerlo prima aggiungere ai preferiti e poi avviarlo.
\newline
\textbf{\emph Versione 1}\\
\textbf{Aspetti negativi riscontrati:}
\begin{itemize}[label=-]

\item difficoltà nel capire a cosa si riferisce il range di prezzo (non per tutti gli utenti).
\item mancanza di una mappa (alcuni utenti cliccavano sulla foto aspettandosi una mappa).
\item manca il rating dell’itinerario.
\item mancano le recensioni.

\end{itemize}

\textbf{Aspetti positivi riscontrati:}
\begin{itemize}[label=-]

\item pulsante aggiungi ai preferiti perfettamente identificato e usato da tutti correttamente
\item pulsante avvia ben posizionato e trovato da tutti senza problemi.
\item descrizione, e icone ben comprese (il prezzo viene compreso dalla maggior parte delle persone ma non dalla totalità).

\end{itemize}

\textbf{Risoluzioni:}
\begin{itemize}[label=-]

\item aggiunte recensioni in fondo alla pagina
\item aggiunta mappa al posto dell’immagine

\end{itemize}

\textbf{\emph Versione 2}\\
\textbf{Aspetti negativi riscontrati:}
\newline\newline
\textbf{Aspetti positivi riscontrati:}
\begin{itemize}[label=-]

\item recensioni identificate e comprese da tutti
\item icona rating ben compresa da tutti
\item mappa ben compresa da tutti

\end{itemize}

\textbf{\emph Note:} Il range di prezzo è stato lasciato anche nella seconda iterazione per continuare a testare la sua comprensione da parte degli utenti, ma ha riportato lo stesso risultato della prima iterazione motivo per il quale abbiamo deciso di rimuoverlo dalla pagina di dettaglio, in quanto gli utenti non erano certi del suo significato. \\


%================= RECENSIONE ============================

\subsection{Recensione}
\textbf{\emph Scenario}\\
Immagina di aver già seguito in passato un itinerario con quest'app.\\
Finalmente l'hai seguito e ti sei fatto un'idea che vuoi condividere con le altre
persone.\\
Vuoi aggiungere i dettagli sulla tua esperienza, magari condividendo foto scattate
durante il tragitto.
\newline
\textbf{\emph Versione 1}\\
\textbf{Aspetti negativi riscontrati:}
\begin{itemize}[label=-]

\item l'utente vorrebbe poter rivedere delle info riguardo l'itinerario di interesse
\item hamburger menù non cliccabile
\item icone foto non cliccabili

\end{itemize}

\textbf{Aspetti positivi riscontrati:}
\begin{itemize}[label=-]

\item il task è stato chiaramente identificato

\end{itemize}

\textbf{\emph Versione 2}\\
\textbf{Aspetti negativi riscontrati:}
\begin{itemize}[label=-]

\item non sapevano quando avevano visitato l'itinerario

\end{itemize}

\textbf{Aspetti positivi riscontrati:}
\begin{itemize}[label=-]

\item pulsante per fare la recensione ben identificato

\end{itemize}

\textbf{\emph Versione 3}\\
\textbf{Aspetti negativi riscontrati:}
\newline\newline

\textbf{Aspetti positivi riscontrati:}
\begin{itemize}[label=-]

\item riescono a capire correttamente quando avevano visitato l'itinerario

\end{itemize}

%================= NUOVO ITINERARIO ============================

\subsection{Inserisci Nuovo Itinerario}
\textbf{\emph Scenario}\\
Immagina di voler aggiungere un nuovo itinerario all’interno dell’app.\\
Hai già aperto la schermata di aggiunta, per cui vuoi specificare il nome dell’itinerario, ad esempio Roma In-Tour, una breve descrizione e informazioni riguardo la durata dell’intero tour.\\
Le tappe che vuoi visitare sono colosseo e circo massimo.
\newline
\textbf{\emph Versione 1}\\
\textbf{Aspetti negativi riscontrati:}
\begin{itemize}[label=-]

\item il tasto in basso con scritto “Fatto” (che indicava la fine dell’inserimento delle
informazioni e quindi l’aggiunta effettiva sulla piattaforma) è stato frainteso
dall’utente.
\item l’utente ha frainteso il campo “Nome” (che rappresentava il nome dell’intero
itinerario) con il Nome delle tappe
\item Anche se non ci sono stati problemi riguardo la comprensione del riepilogo,
l’utente è apparso infastidito dalla presenza di questo riepilogo.

\end{itemize}

\textbf{Aspetti positivi riscontrati:}
\begin{itemize}[label=-]

\item la scelta di impostare la selezione delle tappe, mostrando queste da subito
una sotto l’altra, si è rivelata una buona idea essendo molto simile all’effetto
che si ha su google Maps, infatti, non ci sono stati problemi nella ricerca.

\end{itemize}

\textbf{Risoluzioni:}
\begin{itemize}[label=-]

\item Aggiunto Floating Action Button con classica icona “done”

\end{itemize}

\textbf{\emph Versione 2}\\
\textbf{Aspetti negativi riscontrati:}
\begin{itemize}[label=-]

\item persiste il problema del Nome dell’itinerario scambiato con il Nome della
tappa che si vuole visitare
\item L’utente si è mostrato confuso quando, dopo aver premuto il tasto per aggiungere l’itinerario, gli si è mostrato un riepilogo

\end{itemize}

\textbf{Aspetti positivi riscontrati:}
\begin{itemize}[label=-]

\item Il pulsante sostituito nella fase 1 funziona bene. La classica icona “done” ha
dato l’idea all’utente che l’inserimento fosse terminato e che quindi poteva
procedere con l’inserimento.

\end{itemize}

\textbf{Risoluzioni:}
\begin{itemize}[label=-]
\item Per evitare fraintendimenti con il campo “NOME”, i campi sono stati accompagnati da
un suggerimento che appare nella textfield che dice “Il nome che vuoi dare
all’itinerario” (nel caso ovviamente del campo Nome).
\item La label del campo NOME è stata sostituita con Titolo
\item Pensandoci bene la schermata in sè mostra già una sorta di riepilogo (anche
minimale a livello di quantità di informazioni) per cui è stato deciso di rimuovere la
pagina di riepilogo e semplicemente chiedere la classica conferma.

\end{itemize}

\textbf{\emph Versione 3}\\
\textbf{Aspetti negativi riscontrati:}
\newline\newline

\textbf{Aspetti positivi riscontrati:}
\begin{itemize}[label=-]

\item La modifica di Nome-> TITOLO + l’aiuto che mostriamo in ogni textfield hanno
annullato praticamente ogni dubbio riguardo la semantica del campo!!
\item La rimozione del riepilogo, oltre a non confondere l’utente, ha anche evitato
che quest’ultimo si infastidisse (cosa successa in intervista 1-versione 1)

\end{itemize}

%\subsection{Iterazioni e modifiche apportate}
%
%In questa sezione vengono descritti gli aspetti negativi e positivi riscontrati nelle varie versioni e le modifiche apportate per risolvere i problemi sollevati nelle precedenti.
%\subsubsection{Versione 1}
%
%\begin{itemize}
%
%\item \textbf{Navigazione:} I pin non sono cliccabili, pulsante posizione è fuorviante come pure la freccia in basso.
%
%\item \textbf{Ricerca con filtri:} Alcuni trovano di scarsa utilità il pulsante \emph{Cerca} e vorrebbero far partire l'itinerario dalla tastiera. Tuttavia il pulsante filtraggio è ben posizionato ed identificabile, la barra di ricerca immediatamente identificata e le informazioni esterne ben comprese, come pure il pulsante \emph{Aggiungi ai preferiti}.
%
%\item \textbf{Dettaglio itinerario:} Alcuni utenti hanno difficoltà nel capire a cosa si riferisce il range di prezzo, manca il rating dell’itinerario e
%le recensioni. Tuttavia il pulsante \emph{Aggiungi ai preferiti} è stato perfettamente identificato, come pure il pulsante \emph{Avvia}, la descrizione e tutte le icone ad eccezione di quella del prezzo.
%
%\item \textbf{Recensione itinerario:} L'utente vorrebbe essere in grado di avere informazioni riguardo l'itinerario di interesse. Inoltre l'hamburger menù non è cliccabile, come pure le icone per aggiungere le foto. Tutte le componenti sono state immediatamente identificate.
%
%\end{itemize}
%
%\subsubsection{Versione 2}
%
%\begin{itemize}
%
%\item \textbf{Navigazione:} Ora i pin sono cliccabili e portano alla tappa cliccata, è stata aggiunta una figura circolare blu che indica la posizione attuale. È stata sostituita la freccia con foglio ed infine su Android è stata aggiunto un Floating Action Button per indicare la posizione, che è stato ben identificato.
%
%\item \textbf{Ricerca con filtri:} È stata aggiunta la possibilità di ricercare tramite l’invio dalla tastiera, trovata molto comoda.
%
%\item \textbf{Dettaglio itinerario:} Sono state aggiunte le recensioni in fondo alla pagina che sono state ben identificate da tutti gli utenti.
%
%\item \textbf{Recensione itinerario:} Sono stati resi cliccabili i pulsanti ed è stato aggiunto il pulsante \emph{Recensisci}, perfettamente identificato da tutti gli utenti. Tuttavia molti utenti lamentano la mancanza di un'informazione che gli dica quando è stato visitato l'itinerario.
%
%\end{itemize}
%
%\subsubsection{Versione 3}
%
%\begin{itemize}
%
%\item \textbf{Recensione itinerario:} È stata aggiunta la data di visita che è stata chiaramente identificata da tutti gli utenti, inoltre la recensione è possibile solo se l'itinerario è stato visitato dato che non vengono mostrati altrimenti, comportamento condiviso da tutti.
%
%\end{itemize}