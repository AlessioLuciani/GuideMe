Successivamente sono stati intervistati dei candidati previsti dal target scelto ed è risultato che l'idea proposta non rifletteva i need delle persone. Sono emersi i seguenti need:
\begin{itemize}
	\item La totalità degli intervistati ha affermato di non voler usufruire di guide professioniste, principalmente per una questione economica.
	\item Gli utenti hanno confermato che preferiscono non dover perdere tempo nel ricercare i luoghi da visitare per creare un itinerario. Inoltre alcuni intervistati affermano di trovare i monumenti da visitare come singoli, ma in quanto tali è difficile organizzarli in un itinerario.
	\item È stato confermato che spesso nell'organizzazione degli itinerari viene mal gestito il tempo a disposizione per le visite.
\end{itemize}
Dopo aver analizzato le interviste, abbiamo capito che uno dei punti centrali della nostra idea non è in realtà necessità del target considerato. Quindi abbiamo deciso di modificare l'app escludendo le guide e sostituendole con gli utenti stessi: si passa a una nuova idea che consente agli utenti di consultare itinerari  pubblicati da altri utenti che già hanno fatto quel particolare itinerario in una delle loro esperienze di visita. \\
L'idea è che così facendo siamo in grado di mantenere la fruizione degli itinerari da parte degli utenti, rendendola però totalmente gratuita. \\
Inoltre per mantenere itinerari di qualità è previsto un sistema di recensioni che aiuti gli utenti: coloro che hanno già seguito l'itinerario in passato lasciano un feedback sulla base dell'esperienza che hanno avuto, consentendo agli altri di farsi un'idea. Tramite queste informazioni si rende possibile realizzare un ranking degli itinerari disponibili in modo da facilitare l'utente nella scelta.\\
% questionario pilota
%questionario
%identificazione tasks
%storyboard
%prototyping (loop)



