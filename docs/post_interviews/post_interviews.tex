Successivamente sono stati intervistati dei candidati secondo il target scelto ed è risultato che l'idea proposta non rifletteva del tutto i need precedentemente evidenziati. \\ Sono emersi i seguenti need:
\begin{itemize}
	\item La totalità degli intervistati afferma di non voler usufruire di guide professioniste, principalmente per una questione economica.
	\item Gli utenti confermano che preferiscono non dover perdere tempo nel ricercare i luoghi da visitare per creare un itinerario. Inoltre alcuni hanno affer di riuscire a trovare i monumenti da visitare singolarmente, ma hanno difficoltà ad organizzarli in un itinerario.
	\item Spesso alcuni hanno problemi nella gestione del tempo a disposizione per le visite.
\end{itemize}
Analizzando le interviste abbiamo scoperto che il problema della gestione del tempo era realmente comune a molte persone. Tuttavia abbiamo anche scoperto che il need di avere una guida durante la fruizione dell'itinerario, colonna portante della nostra idea, non
è in realtà percepito come tale dal target considerato. \\

Abbiamo quindi deciso di modificare l'idea dell'app. Questa non prevederà più l'integrazione di guide che verranno invece sostituite dagli stessi utenti: essi potranno consultare gli itinerari pubblicati da altri utenti che li hanno precedentemente seguiti. \\

L'idea è che così facendo rendiamo sempre possibile la fruizione degli itinerari da parte degli utenti, rendendola però totalmente gratuita. \\

Inoltre per mantenere itinerari di qualità è previsto un sistema di recensioni che aiuti gli utenti: coloro che hanno già seguito l'itinerario in passato lasciano un feedback sulla base dell'esperienza che hanno avuto, consentendo agli altri di farsi un'idea. Inoltre, grazie a queste informazioni, è possibile realizzare un ranking degli itinerari disponibili in modo da facilitare l'utente nella scelta.\\



